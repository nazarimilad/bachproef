%%=============================================================================
%% Samenvatting
%%=============================================================================

% TODO: De "abstract" of samenvatting is een kernachtige (~ 1 blz. voor een
% thesis) synthese van het document.
%
% Deze aspecten moeten zeker aan bod komen:
% - Context: waarom is dit werk belangrijk?
% - Nood: waarom moest dit onderzocht worden?
% - Taak: wat heb je precies gedaan?
% - Object: wat staat in dit document geschreven?
% - Resultaat: wat was het resultaat?
% - Conclusie: wat is/zijn de belangrijkste conclusie(s)?
% - Perspectief: blijven er nog vragen open die in de toekomst nog kunnen
%    onderzocht worden? Wat is een mogelijk vervolg voor jouw onderzoek?
%
% LET OP! Een samenvatting is GEEN voorwoord!

%%---------- Nederlandse samenvatting -----------------------------------------
%
% TODO: Als je je bachelorproef in het Engels schrijft, moet je eerst een
% Nederlandse samenvatting invoegen. Haal daarvoor onderstaande code uit
% commentaar.
% Wie zijn bachelorproef in het Nederlands schrijft, kan dit negeren, de inhoud
% wordt niet in het document ingevoegd.

\IfLanguageName{english}{%
\selectlanguage{dutch}
\chapter*{Samenvatting}

\selectlanguage{english}
}{}

%%---------- Samenvatting -----------------------------------------------------
% De samenvatting in de hoofdtaal van het document

\chapter*{\IfLanguageName{dutch}{Samenvatting}{Abstract}}

Dit onderzoek kan dienen om bedrijfsnetwerken beter te beheren en beveiligen tegen interne of externe gevaren met behulp van Cisco Identity Services Engine. Dit omdat bedrijfsnetwerken steeds meer nood hebben aan een network access control product die netwerken beter kunnen beheren en beveiligen. Dit onderzoek focust zich voornamelijk op implementatie en evaluatie van Cisco Identity Services Engine met use cases 'Port-based access control', 'Policy-based access control' en 'Tread-Centric access control'. 
\newline
\newline
Daarnaast worden de resultaten van de enquête tijdens dit onderzoek geanalyseerd die op het einde mee verwerkt zijn met de evaluatie van Cisco Identity Services Engine en zijn use cases. In de literatuurstudie zijn twee andere network access control producten mee verwerkt om aan te tonen dat integratie van deze network access control producten ook mogelijk zijn om een netwerk robuster te maken. Bij de analyses wordt een netwerk geanalyseerd voor integratie van de use cases en eens na de integratie van de use cases. Uit deze twee vergelijkingen werd een evaluatie opgesteld. In dit geschrift vindt u een inleiding tot het onderwerp dat verwerkt is in de literatuurstudie. Daarnaast vindt u de resultaten van de uitgevoerde testen en van de enquête in Hoofdstuk \ref{ch:Resultaten}. 
\newline
\newline
Uit dit onderzoek blijkt dat integratie van Cisco Identity Services Engine de vruchten plukt op beheersbaarheid en beveiliging, dit wordt ook bevestigd in de enquête resultaten. Vervolgens komt uit de enquête naar voor dat 'Identity-based access control' voor vakspecialisten de belangrijkste use case is binnen het Cisco Identity Services Engine product. Toekomstig onderzoek kan over 'Identity-based access control' uitgevoerd worden dat de voordelen van deze use case naar boven brengt.
